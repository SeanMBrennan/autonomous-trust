When the Internet was young (ARPA-net and into the BBS-era), participants were necessarily specialists, and existing social controls allowed for fine-grained trust calibration.
The Internet rapidly advanced well beyond that state, but we continue to act as if human social techniques -- evolved for relatively small groups -- are adequate for an extremely large-scale environment.
In the Cloud, we run proprietary processes with highly sensitive data on someone else's hardware and struggle with the opposing demands to maintain control of access yet be productive in a timely manner.
Zero Trust may be just a re-packaging of well-known system security best-practices (individualized credentials, least privilege access, time-limited sessions) but it is indicative of the growing network security crisis that is already upon us --- and that our best answers are over thirty years old.
Clearly the current cybersecurity paradigm is in trouble: ever-increasing restrictions are complex and anti-scalable, and have but one logical end-point - total lockdown of all resources.
This is the opposite of what we want.
\\[10pt]
What do we want, ideally?
In very general terms, as an owner of certain resources, I want to only give access to the extent that I benefit, no more, no less.
As a user, I also want to benefit from that access, and, aware or it or not, I have value to offer the resource-owner.
Whether client-server or peer-to-peer, every external computation or data-exchange is an implied contract, with risks as well as rewards, along with an implicit level of trust.
So let's make all that \emph{explicit}.

Our premise is that system knowledge -- particularly security context -- is local, situational, and ever changing.
It is best then if control, detection and response are also local.
It follows that we must embrace automation to recreate trust structures for data-sharing.
Like many other successful computing paradigms (genetic algorithms, swarm optimization, neural networks), a solution to this problem becomes apparent through modeling biological processes, in this case, human trust.
What do human trust relationships look like?
\\Generically, they:
\begin{enumerate}
	\item are \textbf{scalable}, multilayered in small groups, allowing a broad but shallow hierarchy, most clearly in military, government or corporate organizations; \label{scale}
	\item share \textbf{language}, often exclusive to a community -- linguistic constructs act as in-group signifiers in addition to means for more efficient knowledge-transfer; \label{language}
	\item share \textbf{goals}, concerns, beliefs or mission; \label{mission}
	\item require \textbf{consistent identity} (preferably immutable) -- naturally accomplished by face-to-face interactions; \label{identity}
	\item are \textbf{reputation-based}, using dynamic evidence of intent -- a lifetime to build, seconds to destroy; \label{reputation}
	\item are \textbf{optimized} in the case of strangers interacting where reputation information is sparse (e.g.\ two-strikes rule); \label{optimized}
	\item are \textbf{prioritized}, utilizing shortcuts, frequently in the form of social roles, since tracking trust relationships tends to be high cognitive load. \label{priorities}
\end{enumerate}
We will justify our breakdown of trust relationships into these seven facets in section~\ref{subsec:trust}.
\\[10pt]
Machine trust would be built with similar characteristics (respectively):
\begin{enumerate}
	\item limited direct communication among peers, but with hierarchical network traversal;
	\item a domain-specific markup language descended from an ur-language, easily extended -- the language a machine implements indicates the groups it participates in;
	\item a strict negotiation protocol, reaching accord in agreed goals;
	\item a crypto-ledger for identity, shared directly among cohorts only but accessible as sub-chains anywhere;
	\item transaction outcome evaluation/scoring, shared and self-immutable -- machine reputation is both public knowledge yet privately evaluated;
	\item allows onboarding and reputation bootstrapping through game-theoretic strategies;
	\item assuming scalability (\#\ref{scale}) is optimized, various prioritization strategies (gullibility, cynicism) can be employed to balance opportunity versus security in groupings that include strangers.
\end{enumerate}
Note that the machine characteristics above directly implement the corresponding aspects of the human trust model.
This reveals our founding assumption: that human trust mechanisms, developed over at least one hundred thousand years, is likely to be more effective than anything we can design.
This is detailed in section~\ref{sec:implementation}.


\subsection{Concept}\label{subsec:concept}

\textbf{\projectName}is TekFive's high-trust cooperative computing concept --- a data messaging framework that allows for dynamic composibility, requesting and serving encrypted data \textit{only} with trusted peers and only to the extent of that fine-grained trust.
Said trust is dynamically evaluated in real time to rapidly eliminate incoming threats and even reclassify existing peers as their behavior changes and thus protect resources.
Increased risk requires a greater trust threshold.
An autonomous agent using this framework can adaptively: 1) refuse communications from severely untrusted peers, conserving bandwidth; 2) communicate with but refuse computation services to faintly trusted peers, protecting CPU time; 3) offer services but refuse data-sharing to moderately trusted peers, protecting data; \textit{and} 4) offer data-sharing to well trusted peers; all with a configurable gradient of access at every level, and all within the same application.

In more concrete terms, \projectName is an operations framework for a vast distributed system that dynamically composes numerous individual microservices into a coherent application on-demand -- with security at its core.
We follow the Unix philosophy: do one thing well, work together, use a universal (text) interface; yet implemented such that each microservice can choose its level of participation.


To illustrate this concept in detail, we will describe a use-case involving secure data-sharing among several entities.
Within this example, we discuss the above seven system facets of communication, language, negotiation, identity, reputation, optimization and prioritization.
Orthogonal to, and cutting through, these conceptual divisions, yet also of critical importance, are operational concerns such as sub-system management and debugging, service discovery, dynamic name-to-address mapping, network resiliency and message reliability.
